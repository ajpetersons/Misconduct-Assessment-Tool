%%
% This file is modified based on the LAYOUT_E.TEX provided in REFMAN package. You can find the original REFMAN package at https://ctan.org/pkg/refman. This document only uses the LAYOUT_E.TEX as a template.

%%
% LAYOUT_E.TEX - Short description of REFMAN.CLS
%                                       99-03-20
%
%  Updated for REFMAN.CLS (LaTeX2e)
%
\documentclass[twoside,a4paper]{refart}
\usepackage{makeidx}
\usepackage{ifthen}
% ifthen wird vom Bild von N.Beebe gebraucht!
\usepackage{url}
\usepackage{graphicx}
\usepackage{xcolor}
\usepackage{listings}
\usepackage{csquotes}
\lstset{basicstyle=\ttfamily,
  showstringspaces=false,
  commentstyle=\color{red},
  keywordstyle=\color{blue}
}

\title{Misconduct Detection Project(MDP) Tool\\
Installation Manual}
\author{Yucheng Xie \\
\today \\
Supervisor: Dr. Kyriakos Kalorkoti\\
Master of Science\\
School of Informatics\\
University of Edinburgh\\
Software Version: 1.0}

\date{}
\emergencystretch1em  %

\pagestyle{myfootings}
\markboth{MDP Tool Installation Manual}%
         {MDP Tool Installation Manual}

\makeindex 

\setcounter{tocdepth}{2}

\begin{document}

\maketitle

\begin{abstract}
This document will help users install the misconduct detection project (MDP) tool on their Windows, Linux or MacOS environment. Please read section \ref{sec:intr} first before you installing the tool no matter which environment you are using.
\end{abstract}

\tableofcontents

\newpage


%%%%%%%%%%%%%%%%%%%%%%%%%%%%%%%%%%%%%%%%%%%%%%%%%%%%%%%%%%%%%%%%%%%%

\section{Introduction} \label{sec:intr}

\subsection{Before Installing} \label{subsec:befo}
The misconduct detection project (MDP) tool\footnote{\url{https://github.com/Weak-Chicken/misconduct_detection_project}} is designed to investigate code segments misconduct information.

This software will be finished in a few iterations. And it has not been finished at the current stage.

\subsection{Which section to use}
To install this tool, the following packages/softwares/tools will be installed and used: Conda, Java SE Runtime Environment(JRE), Python, Django, Git. To help our user choose the proper installation guide, please read following criteria and go to corresponding sections:

\begin{itemize}


\item If you have not installed any of above listed packages/softwares/tools and do not plan to use any one of them; Or you can accept above packages/softwares/tools to be installed in default settings, please go to \seealso{Section \ref{sec:quic}} Section \ref{sec:quic} to begin your installation. \textbf{NOTICE:} If you have any software or program which is relying on above listed packages/softwares/tools, their dependencies might be damaged because the settings of above listed packages/softwares/tools will be reset. \textbf{DO NOT} use this option if you have any software relying on above listed packages/softwares/tools.

\item If you are using any of above listed packages/softwares/tools; Or you are not happy with the default settings of these packages/softwares/tools, please first read the \seealso{Section \ref{subsec:prer}} section \ref{subsec:prer} to make sure your environment meets the pre-requirements. Then, please go to \seealso{Section \ref{sec:step}} Section \ref{sec:step} to begin your installation. \textbf{NOTICE:} This option may require higher skill level than the first option. Please only use this option if it is necessary to control the installation process.

\item Specially, if you are a developer and going to develop next iteration of this tool, please first read the \seealso{Section \ref{subsec:prer}} section \ref{subsec:prer} to make sure your environment meets the pre-requirements. Then, please go to \seealso{Section \ref{sec:late}} Section \ref{sec:late} to begin your installation. \textbf{NOTICE:} Please \textbf{DO NOT} use above convenient installation options. As a developer, it is important to know what is happening when installing the tool.

\end{itemize}

\subsection{Pre-requirements} \label{subsec:prer}
Before continuing on section \ref{sec:step} or section \ref{sec:late}, please make sure your environment has installed following softwares (regardless which operating system you are using):

\begin{itemize}

\item Anaconda/miniconda. If you choose Anaconda, you will need 5.2 or newer. If you choose miniconda, you will need 4.5.4 or newer.

\item Java SE Runtime Environment(JRE) 1.8.0$\_$181 or newer.

\end{itemize}

If you have installed above softwares, you are ready to install this tool. Please go to section \ref{sec:step} or section \ref{sec:late} to start installation.

For those users who would like to know more about the requirements, more explanations are provided:

\begin{itemize}

\item Virtual environment for python. Here this manual chooses Anaconda/miniconda as the virtual environment. If you are advanced user or project developer, please feel free to use other virtual environments. However, please change commands in the section \ref{sec:step} or section \ref{sec:late} accordingly when you install the tool. For users who decide to follow this manual and use Anaconda/miniconda, please make sure the version of Anaconda is 5.2 or newer. Or the version of miniconda is 4.5.4 or newer.

\item Java SE Runtime Environment(JRE). Please be reminded that this tool does \textbf{NOT} need a Java SE Development Kit(JDK) (Of course installing JDK can run this tool, but this tool does not require JDK). The JRE version should be 1.8.0$\_$181 or newer. This will be used to boot JPlag package. If later developer decide to exclude JPlag from this tool, JRE will no longer be needed.

\end{itemize}


\section{Quick Installation Guide} \label{sec:quic}
Please notice that after finishing this installation guide, the following extra softwares will be installed on your computer:

\begin{itemize}

\item Miniconda

\item JRE

\end{itemize}

\subsection{Linux} \label{quic_subsec:linux}

Go to \url{https://github.com/Weak-Chicken/misconduct_detection_project/blob/master/documents/Quick%20Start/quick_start.md} and follow the instructions.

\subsection{Other Systems} \label{quic_subsec:others}
Currently, one-step installation script is not available for other systems. Please check section \ref{sec:step} for installation guide. \seealso{Section \ref{sec:step}}

\section{Step by Step Installation Guide} \label{sec:step}
This Step by Step Installation Guide will suppose you have installed Anaconda/Miniconda and JRE and continues from there.

\subsection{Linux and MacOS} \label{step_subsec:linux_mac}
Please do not directly copy the following commands and run them one by one. The following commands are only used to demonstrate how this thing is working. You need to change them according to your environment. For example, \enquote{$\$conda\_root$} in the following represents the conda installation path rather than a folder called \enquote{$\$conda\_root$}. 

\begin{lstlisting}[language=bash]
git clone https://github.com/Weak-Chicken/miscon
duct_detection_project misconduct_detection_proj
ect/

# Use "conda install git" if you have not installed
# git in your environment.

cd "misconduct_detection_project/documents/
quick start/"

conda env create -f environment.yml

source activate MDP

cd $conda_root/envs/MDP

mkdir -p ./etc/conda/activate.d
mkdir -p ./etc/conda/deactivate.d

echo -e '#!/bin/sh\n' >>
 ./etc/conda/activate.d/env_vars.sh
echo "export SECRET_KEY=changeThisIfYouNeedToDeploy" >>
 ./etc/conda/activate.d/env_vars.sh
echo -e '#!/bin/sh\n' >>
 ./etc/conda/deactivate.d/env_vars.sh
echo 'unset SECRET_KEY' >>
 ./etc/conda/deactivate.d/env_vars.sh

export SECRET_KEY=changeThisIfYouNeedToDeploy
 
cd $install_pos/misconduct_detection_project

python manage.py runserver

# Press ctrl + c to stop it once

python manage.py migrate

python manage.py runserver
\end{lstlisting}

\subsection{Windows} \label{step_subsec:win}
Under Windows, you first need to open the \enquote{Anaconda Prompt} to continue the following operations.

\begin{lstlisting}[language=bash]
git clone https://github.com/Weak-Chicken/miscon
duct_detection_project misconduct_detection_proj
ect/

# Use "conda install git" if you have not installed
# git in your environment.

cd "misconduct_detection_project/documents/
quick start/"

conda env create -f environment.yml

conda activate MDP

cd $conda_root\envs\MDP

mkdir .\etc\conda\activate.d
mkdir .\etc\conda\deactivate.d

@echo "set SECRET_KEY=changeThisIfYouNeedToDeploy">>
 .\etc\conda\activate.d\env_vars.bat
@echo "set SECRET_KEY="  >>
 .\etc\conda\deactivate.d\env_vars.bat
 
set SECRET_KEY="changeThisIfYouNeedToDeploy"

cd $install_pos/misconduct_detection_project

python manage.py runserver

# Press ctrl + c to stop it once

python manage.py migrate

python manage.py runserver

\end{lstlisting}

\section{Later Developer Installation Guide} \label{sec:late}
For developers' installation guide, we will use Linux as our demonstration environment. If you need to run this on Mac or Windows, please change it accordingly.

This guide is based on \url{https://github.com/Weak-Chicken/misconduct_detection_project/blob/master/documents/Quick%20Start/install_MDP_tool_without_installing_conda.sh}. You can download it from the github page.

This script can be divided into five parts.

$\ $

Part 1:
\begin{lstlisting}[language=bash]
#!/usr/bin/env bash
echo "name: MDP" >> environment.yml
echo "channels:" >> environment.yml
echo "  - defaults" >> environment.yml
echo "dependencies:" >> environment.yml
echo "  - certifi=2018.4.16=py37_0" >> environment.yml
echo "  - pip=10.0.1=py37_0" >> environment.yml
echo "  - python=3.7.0" >> environment.yml
echo "  - setuptools=39.2.0=py37_0" >> environment.yml
echo "  - wheel=0.31.1=py37_0" >> environment.yml
echo "  - git=2.17.0" >> environment.yml
echo "  - pip:" >> environment.yml
echo "    - beautifulsoup4==4.6.1" >> environment.yml
echo "    - django==2.0.7" >> environment.yml
echo "    - pytz==2018.5" >> environment.yml
echo "prefix: C:\Users\YuchengXie\Anaconda3\envs\MDP" >>
 environment.yml
echo "" >> environment.yml
\end{lstlisting}

This part is used to create requirement file. To ensure our script will not go wrong, we decide to embed git inside our conda environment. Therefore, to use git, we need to have the environment. However, to install environment, we need the requirement file from git. To resolve this dilemma, we embedded requirement file in the installation script so that we can create our environment before using git clone.

$\ $

Part 2:
\begin{lstlisting}[language=bash]
conda_root=${1:-$HOME/miniconda3/envs/MDP}
install_pos=${2:-$(pwd)}

echo "Your conda root is:"
echo $conda_root
echo "Your installation path is:"
echo $install_pos
\end{lstlisting}

This part is used to set the paths and show them to the user.

$\ $

Part 3:
\begin{lstlisting}[language=bash]
mkdir -p $install_pos

conda env create -f environment.yml
source activate MDP
rm environment.yml

cd $install_pos

git clone https://github.com/Weak-Chicken/misconduct
_detection_project misconduct_detection_project/
\end{lstlisting}

This part is used to create conda environment and clone git repository to local.

$\ $

Part 4:
\begin{lstlisting}[language=bash]
cd $conda_root

mkdir -p ./etc/conda/activate.d
mkdir -p ./etc/conda/deactivate.d

echo -e '#!/bin/sh\n' >>
 ./etc/conda/activate.d/env_vars.sh
echo "export SECRET_KEY=changeThisIfYouNeedToDeploy" >>
 ./etc/conda/activate.d/env_vars.sh
echo -e '#!/bin/sh\n' >>
 ./etc/conda/deactivate.d/env_vars.sh
echo 'unset SECRET_KEY' >>
 ./etc/conda/deactivate.d/env_vars.sh

source deactivate MDP
source activate MDP
\end{lstlisting}

The part 4 is setting correct environment variables to the virtual environment. As explained in the dissertation, we provide our secret key as environment variable to keep it safe. Although this one is not safe since we have uploaded it online. It is worth mentioning that when the tool is deployed, the secret key here must be changed.

$\ $

Part 5:
\begin{lstlisting}[language=bash]
cd $install_pos/misconduct_detection_project

timeout 3 python manage.py runserver
python manage.py migrate

echo "=============================================="
echo "Installation finished!"
echo "use 'source activate MDP' to activate the vir
tual environment"
echo "cd into the tool folder, e.g. 'cd misconduct_
detection_project'"
echo "and use 'python manage.py runserver' to begin"
\end{lstlisting}

The final part here will perform migration and give the hint for users to begin. Here we need to first execute \enquote{python manage.py runserver} and then execute \enquote{python manage.py migrate} to finish the migration part. At current stage, we have not used database functions, thus migration here will create an empty database in the folder.

$\ $

In conclusion, to run this tool, you need to do following steps:
\begin{enumerate}
\item Install Anaconda/Miniconda and JRE
\item Install virtual environment through the requirement file.
\item Set secret key environment variable.
\item Perform database migration. Which will create an empty database.
\end{enumerate}

%\clearpage
%
%\section*{Appendix}
%\addcontentsline{toc}{section}{Appendix}
%
%\appendix
%
%\section{Install Conda}

\printindex

\end{document}
